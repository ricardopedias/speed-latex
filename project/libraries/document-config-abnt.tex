% ++++++++++++++++++++++++++++++++++++++++++++++++++++++++++++++++++++++++
% Autor              : Ricardo Pereira Dias <contato@ricardopdias.com.br>
% Data de criação    : 29 de Janeiro de 2019
% Referência         : https://github.com/abntex/abntex2/blob/master/tex/latex/abntex2/abntex2.cls
% ++++++++++++++++++++++++++++++++++++++++++++++++++++++++++++++++++++++++

% =========================================================================
% COR PADRÂO DOS TEXTOS
% =========================================================================
\color{body_text}

% =========================================================================
% TIPO DE FAMILIA PADRÃO
% =========================================================================
% Existem 3 tipos de familias
% roman (serif)              \textrm     \rmfamily
% sans serif                 \textsf     \sffamily
% typewriter (monospace)     \texttt     \ttfamily

\renewcommand{\familydefault}{\sfdefault} % sans
% \renewcommand{\familydefault}{\rmdefault} % roman

% =========================================================================
% CONFIGURAÇÃO DO BACKREF
% =========================================================================
% Usado sem a opção hyperpageref de backref
\renewcommand{\backrefpagesname}{Citado na(s) página(s):~}
% Texto padrão antes do número das páginas
\renewcommand{\backref}{}
% Define os textos da citação
\renewcommand*{\backrefalt}[4]{
    \ifcase #1 %
        Nenhuma citação no texto.%
    \or
        Citado na página #2.%
    \else
        Citado #1 vezes nas páginas #2.%
    \fi%
}%

%=========================================================================
% VARIÁVEIS DO ABNTEX
%=========================================================================

    % Documento
\titulo{\projectName}
\local{\projectCity}
\data{\projectDate\projectVersion}
    % Instituição
\instituicao{\schoolInstitution}
    % Projeto
\tipotrabalho{\projectType}
\autor{\projectAuthor}
\preambulo{\projectPreamble}

% TCC
\orientador{\projectOrienting}
\coorientador{\projectCoorienting}

%=========================================================================
% APARÊNCIA DO PDF FINAL
%=========================================================================

\makeatletter
\hypersetup{
        %pagebackref=true,
        pdftitle={\@title},
        pdfauthor={\@author},
        pdfsubject={\projectPreamble},
        pdfcreator={LaTeX with abnTeX2},
        pdfkeywords={abnt}{latex}{abntex}{abntex2}{artigo científico},
        colorlinks=true,               % false: boxed links; true: colored links
        linkcolor=blue,              % color of internal links
        citecolor=blue,                % color of links to bibliography
        filecolor=magenta,              % color of file links
        urlcolor=blue,
        bookmarksdepth=4
}
\makeatother

%=========================================================================
% CABEÇALHOS E RODAPÉS
%=========================================================================

\makepagestyle{abntheadings}
\makepagestyle{abntchapfirst}

% ------------------------------------------------------------------------
% Cabeçalhos
% ------------------------------------------------------------------------

% Primeira página de um capítulo (abnt é sempre impar)
\makeoddhead{abntchapfirst}
    {\color{header_text}\headerOddLeft}   % esquerda
    {\color{header_text}\headerOddCenter} % centro
    {\color{header_text}\headerOddRight}  % direita

% Primeira página de um capítulo (se form customizado para par)
\makeevenhead{abntchapfirst}
    {\color{header_text}\headerEvenLeft}   % esquerda
    {\color{header_text}\headerEvenCenter} % centro
    {\color{header_text}\headerEvenRight}  % direita

% Página ímpar ou com oneside
\makeoddhead{abntheadings}
    {\color{header_text}\headerOddLeft}   % esquerda
    {\color{header_text}\headerOddCenter} % centro
    {\color{header_text}\headerOddRight}  % direita

% Página par
\makeevenhead{abntheadings}
    {\color{header_text}\headerEvenLeft}   % esquerda
    {\color{header_text}\headerEvenCenter} % centro
    {\color{header_text}\headerEvenRight}  % direita

% ------------------------------------------------------------------------
% Rodapés
% ------------------------------------------------------------------------
% Primeira página de um capítulo
\makeoddfoot{abntchapfirst}
    {\color{footer_text}\footerOddLeft}   % esquerda
    {\color{footer_text}\footerOddCenter} % centro
    {\color{footer_text}\footerOddRight}  % direita

% Página ímpar ou com oneside
\makeoddfoot{abntheadings}
    {\color{footer_text}\footerOddLeft}   % esquerda
    {\color{footer_text}\footerOddCenter} % centro
    {\color{footer_text}\footerOddRight}  % direita

% Página par
\makeevenfoot{abntheadings} % pagina par
    {\color{footer_text}\footerEvenLeft}   % esquerda
    {\color{footer_text}\footerEvenCenter} % centro
    {\color{footer_text}\footerEvenRight}  % direita

% ------------------------------------------------------------------------
% LINHAS SEPERADORAS DE CABEÇALHOS E RODAPÉS
% ------------------------------------------------------------------------
% Cores
\makeheadfootruleprefix{abntheadings}
    {\color{header_rule}}   % cor da linha do cabeçalho
    {\color{footer_rule}}   % cor da linha do rodapé
\makeheadfootruleprefix{abntchapfirst}
    {\color{header_rule}}   % cor da linha do cabeçalho
    {\color{footer_rule}}   % cor da linha do rodapé
% Dimensões
\makeheadrule{abntheadings}
    {\textwidth}        % Abrangência da linha do cabeçalho
    {\headerRuleHeight} % Espessura da linha do cabeçalho
\makefootrule{abntheadings}
    {\textwidth}        % Abrangência da linha do rodapé
    {\footerRuleHeight} % Espessura da linha do rodapé
    {3pt}               % Distancia entre a linha e o rodapé
\makeheadrule{abntchapfirst}
    {\textwidth}        % Abrangência da linha do cabeçalho
    {\headerRuleHeight} % Espessura da linha do cabeçalho
\makefootrule{abntchapfirst}
    {\textwidth}        % Abrangência da linha do rodapé
    {\footerRuleHeight} % Espessura da linha do rodapé
    {3pt}               % Distancia entre a linha e o rodapé

%=========================================================================
% FIGURAS E TABELAS
%=========================================================================
% Posiciona figuras e tabelas no topo da página quando adicionadas sozinhas
% em um página em branco. Ver https://github.com/abntex/abntex2/issues/170
\makeatletter
\setlength{\@fptop}{5pt} % Set distance from top of page to first float
\makeatother

%=========================================================================
% QUADROS E LISTA DE QUADROS
%=========================================================================
% Possibilita criação de Quadros e Lista de quadros.
% Ver https://github.com/abntex/abntex2/issues/176
\newcommand{\quadroname}{Quadro}
\newcommand{\listofquadrosname}{Lista de quadros}
\newfloat[chapter]{quadro}{loq}{\quadroname}
\newlistof{listofquadros}{loq}{\listofquadrosname}
\newlistentry{quadro}{loq}{0}
% configurações para atender às regras da ABNT
\setfloatadjustment{quadro}{\centering}
\counterwithout{quadro}{chapter}
\renewcommand{\cftquadroname}{\quadroname\space}
\renewcommand*{\cftquadroaftersnum}{\hfill--\hfill}
\setfloatlocations{quadro}{hbtp} % Ver https://github.com/abntex/abntex2/issues/176


%=========================================================================
% ESPEÇAMENTOS, ENTRE-LINHAS E PARÁGRAFOS
%=========================================================================

% O tamanho do parágrafo é dado por:
\setlength{\parindent}{1.3cm}

% Controle do espaçamento entre um parágrafo e outro:
\setlength{\parskip}{0.2cm}  % tente também \onelineskip

% Espaçamento simples
% \SingleSpacing

%=========================================================================
% CAPÍTULOS E SEÇÕES
%=========================================================================

\renewcommand{\ABNTEXpartfont}{\color{part_text}\partStyle\partFont}
\renewcommand{\ABNTEXpartfontsize}{\partSize}
\renewcommand{\partnamefont}{\color{part_count}\partAssertionStyle\partAssertionFont\partAssertionSize}
\renewcommand{\partnumfont}{\color{part_count}\partAssertionStyle\partAssertionFont\partAssertionSize}

% Adiciona ponto no final do numero de seções
% \setsecnumformat{\csname the#1\endcsname.\quad}

% http://ctan.math.utah.edu/ctan/tex-archive/macros/latex/contrib/tocloft/tocloft.pdf

% Capítulos
\renewcommand{\ABNTEXchapterfont}{\color{chapter_text}\chapterStyle\chapterFont}
\renewcommand{\ABNTEXchapterfontsize}{\chapterSize}
\renewcommand{\chapnumfont}{\color{chapter_count}\chapterStyle\chapterFont}
% \renewcommand*\thechapter{\arabic{chapter}\sectioningFinalDot} % \Alph, \Roman, \arabic

% Seções
\renewcommand{\ABNTEXsectionfont}{\color{section_text}\sectionStyle\sectionFont}
\renewcommand{\ABNTEXsectionfontsize}{\sectionSize}
% \renewcommand\thesection{\arabic{chapter}.\arabic{section}\sectioningFinalDot} % \Alph, \Roman, \arabic
%\setsecnumformat{\csname\color{section_count}the#1\endcsname\quad}

% Sub-section
\renewcommand{\ABNTEXsubsectionfont}{\color{subsection_text}\subsectionStyle\subsectionFont}
\renewcommand{\ABNTEXsubsectionfontsize}{\subsectionSize}
% \renewcommand{\thesubsection}{\arabic{chapter}.\arabic{section}.\arabic{subsection}\sectioningFinalDot} % \Alph, \Roman, \arabic

% Sub-sub-section
\renewcommand{\ABNTEXsubsubsectionfont}{\color{subsubsection_text}\subsubsectionStyle\subsubsectionFont}
\renewcommand{\ABNTEXsubsubsectionfontsize}{\subsubsectionSize}
% \renewcommand{\thesubsubsection}{\arabic{chapter}.\arabic{section}.\arabic{subsection}.\arabic{subsubsection}\sectioningFinalDot} % \Alph, \Roman, \arabic

\renewcommand{\ABNTEXsubsubsubsectionfont}{\color{subsubsubsection_text}\subsubsubsectionStyle\subsubsubsectionFont}
\renewcommand{\ABNTEXsubsubsubsectionfontsize}{\subsubsectionSize}
% \renewcommand{\theparagraph}{\arabic{chapter}.\arabic{section}.\arabic{subsection}.\arabic{subsubsection}.\arabic{paragraph}\sectioningFinalDot} % \Alph, \Roman, \arabic

% ------------------------------------------------------------------------
% compila o indice
% ------------------------------------------------------------------------
\makeindex
