%====================================================================================================
%----------------------------------------------------------------------------------------------------
% Autor              : Ricardo Pereira Dias <rpdesignerfly@gmail.com>
%----------------------------------------------------------------------------------------------------
% Data de criação    : 29 de Janeiro de 2019
%====================================================================================================

%\usepackage[top=30mm,bottom=25mm,left=25mm,right=20mm]{geometry}
%Margens

\usepackage{lipsum} % para preencher com textos padrões
\usepackage[top=3cm, bottom=2cm, left=3cm, right=2cm]{geometry}
\usepackage{pdfpages}
\usepackage[brazil]{babel}
\usepackage{multicol}
\usepackage{booktabs}     % http://ctan.org/pkg/booktabs
\usepackage{array}        % http://ctan.org/pkg/array
\newcolumntype{M}{>{\centering\arraybackslash}m{\dimexpr.25\linewidth-2\tabcolsep}}


%Pacote de edição de arquivos PDF e de imagem
\usepackage{tikz}

%Configurações Lista de Siglas e Abreviações
\usepackage[nopostdot,style=super,nonumberlist,toc]{glossaries}
\newglossarystyle{modsuper}{
    \glossarystyle{super}
    \renewcommand{\glsgroupskip}{}
}
\makeglossaries

%Pacote para Indetação dos parágrafos iniciais
\usepackage{indentfirst}

%Pacote para alterar espaçamento entre linhas
\usepackage{setspace}
%\singlespacing Para um espaçamento simples
%\onehalfspacing Para um espaçamento de 1,5
%\doublespacing Para um espaçamento duplo

%Pacote para alterar espaçamento entre linhas específicas
\usepackage{leading}
%\leading{15pt}


% para hackear a linha de cabeçalhos e rodapés
\usepackage{xpatch}

%Pacote de Numeração sequencial para figuras, tabelas e quadros
\usepackage{chngcntr}

%Pacote para configurar cores
\usepackage{xcolor}

\usepackage{ulem} % para riscar textos
