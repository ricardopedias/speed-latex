%=========================================================================
% Autor              : Ricardo Pereira Dias <rpdesignerfly@gmail.com>
%-------------------------------------------------------------------------
% Data de criação    : 29 de Janeiro de 2019
% Cores: https://www.overleaf.com/learn/latex/Using_colours_in_LaTeX
%=========================================================================

%=========================================================================
% CAPA e FOLHA DE ROSTO
%=========================================================================

% Projeto de Pesquisa
% TCC
% Artigo
\newcommand{\schoolInstitution}{%
    Universidade do Brasil -- UBr
    \par
    Faculdade de Arquitetura da Informação
    \par
    Programa de Pós-Graduação
}
\newcommand{\projectType}{Projeto de pesquisa}
\newcommand{\projectName}{Modelo Canônico de \projectType{}  com Speed Latex}
\newcommand{\projectCity}{São Carlos - SP}
\newcommand{\projectDate}{2019}
\newcommand{\projectVersion}{, v-1.0.0}
% O preambulo deve conter o tipo do trabalho, o objetivo,
% o nome da instituição e a área de concentração
\newcommand{\projectPreamble}{Modelo canônico de  \projectType{}  em conformidade com as normas ABNT apresentado à comunidade de usuários \LaTeX.}
\newcommand{\projectAuthor}{Ricardo Pereira Dias}
\newcommand{\projectContact}{contato@ricardopdias.com.br}

% TCC
\newcommand{\projectOrientingLabel}{Prof.}%rotulo
\newcommand{\projectOrienting}{Linuz Torvalds}
\newcommand{\projectCoorientingLabel}{Prof.}%rotulo
\newcommand{\projectCoorienting}{Richard Stallman}

% Artigo
\newcommand{\projectForeignName}{Canonical article template in SpeedLatex} % Nome extrangeiro

%=========================================================================
% CABEÇALHO
%=========================================================================
% Aparência
\definecolor{header_rule}{RGB}{ 200, 200, 200 }
\definecolor{header_text}{RGB}{ 200, 200, 200 }
% \newcommand{\headerRuleHeight}{0.4pt}
\newcommand{\headerRuleHeight}{0pt}
% Páginas ímpares
\newcommand{\headerOddLeft}{}
\newcommand{\headerOddCenter}{}
\newcommand{\headerOddRight}{\thepage}
% Páginas Pares
\newcommand{\headerEvenLeft}{\thepage}
\newcommand{\headerEvenCenter}{}
\newcommand{\headerEvenRight}{}


%=========================================================================
% RODAPÉ
%=========================================================================
% Aparência
\definecolor{footer_rule}{RGB}{ 200, 200, 200 }
\definecolor{footer_text}{RGB}{ 200, 200, 200 }
% \newcommand{\footerRuleHeight}{0.4pt}
\newcommand{\footerRuleHeight}{0pt}
% Páginas ímpares
\newcommand{\footerOddLeft}{}
\newcommand{\footerOddCenter}{}
\newcommand{\footerOddRight}{}
% Páginas Pares
\newcommand{\footerEvenLeft}{}
\newcommand{\footerEvenCenter}{}
\newcommand{\footerEvenRight}{}

%=========================================================================
% CORES GERAIS
%=========================================================================
% http://erikasarti.com/html/tabela-cores/
\definecolor{black}{RGB}{0,0,0}
% \definecolor{blue}{RGB}{41,5,195}
\definecolor{blue}{RGB}{2,69,156}
\definecolor{dim_gray}{RGB}{105,105,105}
\definecolor{gray}{RGB}{128,128,128}
\definecolor{dark_gray}{RGB}{169,169,169}
\definecolor{silver}{RGB}{192,192,192}
\definecolor{light_grey}{RGB}{211,211,211}
\definecolor{gainsboro}{RGB}{220,220,220}
\definecolor{white}{RGB}{255,255,255}

\definecolor{success_text}{RGB}{ 43, 180, 54 }
\definecolor{info_text}{RGB}{ 2, 69, 156 }
\definecolor{warning_text}{RGB}{ 255, 158, 0}
\definecolor{danger_text}{RGB}{ 255, 60, 0 }

%=========================================================================
% CAPÍTULOS E SEÇÕES
%=========================================================================

% http://www.tug.dk/FontCatalogue/

% \newcommand{\fontSansOne}{roboto}
% \newcommand{\fontSerifOne}{\sffamily}
% \newcommand{\fontMonoOne}{\sffamily}

% \usepackage{fontspec}
% \setmainfont[Ligatures=TeX]{Georgia}
% \setsansfont[Ligatures=TeX]{Arial}


% Vários exemplos de capitulos
% http://tug.ctan.org/info/MemoirChapStyles/
% https://texblog.org/2012/07/03/fancy-latex-chapter-styles/

% Para formatar todos os niveis de seção
\newcommand{\sectioningTopStyle}{\bfseries}
\newcommand{\sectioningSecStyle}{\bfseries}
\newcommand{\sectioningFinalDot}{} % Para poder adicionar um character no final da numeração


% Partes
% Declaração: Parte 1
\newcommand{\partAssertionFont}{\sffamily}
\newcommand{\partAssertionStyle}{\sectioningTopStyle}
\newcommand{\partAssertionSize}{\large}
% Título da parte
\newcommand{\partFont}{\sffamily}
\newcommand{\partStyle}{\sectioningTopStyle}
\newcommand{\partSize}{\Huge}
\definecolor{part_count}{RGB}{ 2, 69, 156 }
\definecolor{part_text}{RGB}{ 255, 69, 156 }

% Capítulos
% \newcommand{\chapterFont}{\sffamily}
\newcommand{\chapterFont}{\sffamily}
\newcommand{\chapterStyle}{\sectioningTopStyle}
\newcommand{\chapterSize}{\Huge}
\definecolor{chapter_count}{RGB}{ 200, 69, 0 }
\definecolor{chapter_text}{RGB}{ 2, 69, 156 }

% Seções
\newcommand{\sectionFont}{\sffamily}
\newcommand{\sectionStyle}{\sectioningSecStyle}
\newcommand{\sectionSize}{\Large}
\definecolor{section_count}{RGB}{ 200, 69, 50 }
\definecolor{section_text}{RGB}{ 2, 69, 156 }

% Sub Seções
\newcommand{\subsectionFont}{\sffamily}
\newcommand{\subsectionStyle}{\sectioningSecStyle}
\newcommand{\subsectionSize}{\large}
\definecolor{subsection_count}{RGB}{ 200, 100, 50 }
\definecolor{subsection_text}{RGB}{ 2, 69, 156 }

% Sub Sub Seções
\newcommand{\subsubsectionFont}{\sffamily}
\newcommand{\subsubsectionStyle}{\sectioningSecStyle}
\newcommand{\subsubsectionSize}{\normalsize}
\definecolor{subsubsection_count}{RGB}{ 200, 69, 0 }
\definecolor{subsubsection_text}{RGB}{ 2, 69, 156 }

% Sub Sub Sub Seções
\newcommand{\subsubsubsectionFont}{\sffamily}
\newcommand{\subsubsubsectionStyle}{\sectioningSecStyle}
\newcommand{\subsubsubsectionSize}{\normalsize}
\definecolor{subsubsubsection_count}{RGB}{ 200, 69, 0 }
\definecolor{subsubsubsection_text}{RGB}{ 2, 69, 156 }

\definecolor{paragraph_text}{RGB}{ 15, 15, 15 } % parágrafo de leis ou documentos
\definecolor{subparagraph_text}{RGB}{ 50, 50, 50 } % subparágrafo de leis ou documentos

% Corpo de texto
\definecolor{body_text}{RGB}{ 100, 100, 100 }
\definecolor{link_text}{RGB}{ 2, 69, 156 }
