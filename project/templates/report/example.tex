%===================================================================================================
% Autor             : Ricardo Pereira Dias <rpdesignerfly@gmail.com>
% Site              : https://rpdesignerfly.github.io
% Data de criação   : 29 de Janeiro de 2019
% Referencias       : https://pt.overleaf.com/learn
%                     https://texblog.org/code-snippets/beamer-presentation/
% --------------------------------------------------------------------------------------------------
%
% Itálico: \textItalic{meu texto}.
% Negrito: \textBold{meu texto}.
% Enfatizado: \textit{meu \emph{texto} digitado}
% Sublinhado: \textUnderline{meu texto}
% Sublinhado Duplo: \textDouble{meu texto}
% Sublinhado Ondulado: \textWaveUnderline{meu texto}
% Riscado: \textDelete{meu texto}
% Rasurado: \textStriped{meu texto}
%===================================================================================================

\docSection{ Formatação de seções }

\docBodyText{
    \lipsum[1]
}

\docSubSection{ Formatação de seções }

\docBodyText{
    \lipsum[1]
}

\docSubSubSection{ Formatação de seções }

\docBodyText{
    \lipsum[1]
}

% ----------------------------------------------------------------------------------------------------------------------

\docSection{ Formatação de parágrafos }

\docBodyText{
    \lipsum[1]\lipsum[1]
}

\docSection{ Formatação de textos }

\begin{itemize}

    \item \textItalic{Texto talic}
    \item \textBold{Texto bold}
    \item \textSuccess{Texto success}
    \item \textDanger{Texto danger}
    \item \textInfo{Texto info}
    \item \textWarning{Text warning}

    \item \textUnderline{Underline} \textUnderlineDanger{danger} \textUnderlineSuccess{success} \textUnderlineDanger{\textNormal{normal}} \textUnderlineSuccess{\textNormal{normal}}
    \item \textDouble{Underline Double} \textDoubleDanger{danger} \textDoubleSuccess{success} \textDoubleDanger{\textNormal{normal}} \textDoubleSuccess{\textNormal{normal}}
    \item \textWave{Wave} \textWaveDanger{danger} \textWaveSuccess{success} \textWaveDanger{\textNormal{normal}} \textWaveSuccess{\textNormal{normal}}
    \item \textDelete{Delete} \textDeleteDanger{danger} \textDeleteSuccess{success} \textDeleteDanger{\textNormal{normal}} \textDeleteSuccess{\textNormal{normal}}
    \item \textStriped{Spriped} \textStripedDanger{danger} \textStripedSuccess{success} \textStripedDanger{\textNormal{normal}} \textStripedSuccess{\textNormal{normal}}

\end{itemize}




\docSection{ Missão }


\textit{ ( Descreva a missão do projeto ) }

O objetivo do sistema é permitir o comércio de miniaturas e divulgo empresa pela Internet

% --------------------------------------------------------------------------------------------------

\section{Relação de objetivos}

\vskip 0.4cm
{\fontsize{18pt}{18pt} \selectfont \bf Relação de objetivos}
\vskip 0.2cm

\textit{ ( Relacione os objetivos do projeto, relacionados às necessidades de negócio ) }

Os objetivos do sistema web podem ser definidos pelos seguintes itens:

\begin{itemize}
    \item Divulgar a empresa na Internet
    \item Prospectar e intensificar clientes
    \item Permitir o comércio eletrônico de miniaturas
    \item Permitir o acesso aos dados cadastrais dos clientes cadastrados no sistema para ações de marketing
    \item Mostrar informações sobre vendas, tais como, produtos mais vendidos, clientes com maior volume de compras
    \item Conter informações sobre a própria empresa, fotos, descrição e preço das miniaturas disponíveis para venda
\end{itemize}

% --------------------------------------------------------------------------------------------------

\vskip 0.4cm
{\fontsize{18pt}{18pt} \selectfont \bf Público alvo}
\vskip 0.2cm

\textit{ ( Descreva para qual público alvo o sistema estará direcionado ) }

Pessoas de diferentes idades, crianças, jovens e aficcionados por miniaturas

% --------------------------------------------------------------------------------------------------

\vskip 0.4cm
{\fontsize{18pt}{18pt} \selectfont \bf Área de negócio solicitante}
\vskip 0.2cm

\textit{ ( Identifique a área de negócio da empresa que está demandando o projeto. ) }

Gestor do negócio: Departamento de Vendas.

% --------------------------------------------------------------------------------------------------

\vskip 0.4cm
{\fontsize{18pt}{18pt} \selectfont \bf Desenvolvimento do sistema}
\vskip 0.2cm

\textit{ ( Identifique quem será responsável pelo desenvolvimento técnico do projeto, e desenvolvimento interno ou externo. ) }

Empresa: A ser selecionada por Request for Proposal - RFP

% --------------------------------------------------------------------------------------------------

\vskip 0.4cm
{\fontsize{18pt}{18pt} \selectfont \bf Requisitos funcionais}
\vskip 0.2cm

\textit{ ( Relacione os requisitos funcionais do sistema ) }
O sistema deverá:

\begin{itemize}
    \item Ser acessado pela Internet.

\end{itemize}

Apresentar informacões sobre a empresa (Quem somos),
Apresentar informações aos usuários de como utilizar o sistema par
se cadastrar e realizar compras (Utilizando o site).
Possuir os perfis de acesso de visitante, cliente e administrador.
e
negócio./
tes itens:
Permitir aos usuários:
Acessar o sistema pela Internet sem necessidade de instalação de sof
twares em seus computadores.
. Visualizar as miniaturas disponíveis na empresa e dados relacionados
no
lastrados
às mesmas, sem necessidade de se cadastrar no sistema.
Cadastrar-se no sistema sem a obrigatoriedade da realização de pedidos
.

% --------------------------------------------------------------------------------------------------

\vskip 0.4cm
{\fontsize{18pt}{18pt} \selectfont \bf Requisitos não funcionais}
\vskip 0.2cm

\lipsum[1]

% --------------------------------------------------------------------------------------------------

\vskip 0.4cm
{\fontsize{18pt}{18pt} \selectfont \bf Design visual}
\vskip 0.2cm

\lipsum[1]

% --------------------------------------------------------------------------------------------------

\vskip 0.4cm
{\fontsize{18pt}{18pt} \selectfont \bf Restrições de implementação}
\vskip 0.2cm

\lipsum[1]
