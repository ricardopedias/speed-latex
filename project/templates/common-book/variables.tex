%====================================================================================================
%----------------------------------------------------------------------------------------------------
% Autor              : Ricardo Pereira Dias <rpdesignerfly@gmail.com>
%----------------------------------------------------------------------------------------------------
% Data de criação    : 29 de Janeiro de 2019
%====================================================================================================

% Variáveis

\newcommand{\projectType}{Livro Simples}
\newcommand{\projectName}{Modelo de \projectType}
\newcommand{\projectAuthor}{Ricardo Pereira Dias}
\newcommand{\projectContact}{contato@ricardopdias.com.br}
\newcommand{\projectCity}{São Carlos - SP}
\newcommand{\projectDate}{2019}
\newcommand{\projectVersion}{v1.0.0}


%=========================================================================
% CABEÇALHO
%=========================================================================
% Aparência
\definecolor{header_rule}{RGB}{ 200, 200, 200 }
\definecolor{header_text}{RGB}{ 200, 200, 200 }
\newcommand{\headerRuleHeight}{0.3pt}
% Páginas ímpares
\newcommand{\headerOddLeft}{\small\leftmark}
\newcommand{\headerOddCenter}{}
\newcommand{\headerOddRight}{}
% Páginas Pares
\newcommand{\headerEvenLeft}{}
\newcommand{\headerEvenCenter}{}
\newcommand{\headerEvenRight}{\small\rightmark}


%=========================================================================
% RODAPÉ
%=========================================================================
% Aparência
\definecolor{footer_rule}{RGB}{ 200, 200, 200 }
\definecolor{footer_text}{RGB}{ 200, 200, 200 }
\newcommand{\footerRuleHeight}{.3pt}
% Páginas ímpares
\newcommand{\footerOddLeft}{\small\projectName}
\newcommand{\footerOddCenter}{}
\newcommand{\footerOddRight}{\small\thepage}
% Páginas Pares
\newcommand{\footerEvenLeft}{\small\thepage}
\newcommand{\footerEvenCenter}{}
\newcommand{\footerEvenRight}{\small\projectName}

%=========================================================================
% CORES GERAIS
%=========================================================================
% http://erikasarti.com/html/tabela-cores/
\definecolor{black}{RGB}{0,0,0}
% \definecolor{blue}{RGB}{41,5,195}
\definecolor{blue}{RGB}{2,69,156}
\definecolor{dim_gray}{RGB}{105,105,105}
\definecolor{gray}{RGB}{128,128,128}
\definecolor{dark_gray}{RGB}{169,169,169}
\definecolor{silver}{RGB}{192,192,192}
\definecolor{light_grey}{RGB}{211,211,211}
\definecolor{gainsboro}{RGB}{220,220,220}
\definecolor{white}{RGB}{255,255,255}

\definecolor{success_text}{RGB}{ 43, 180, 54 }
\definecolor{info_text}{RGB}{ 10, 70, 160 }
\definecolor{warning_text}{RGB}{ 255, 158, 0}
\definecolor{danger_text}{RGB}{ 255, 60, 0 }

%=========================================================================
% CAPÍTULOS E SEÇÕES
%=========================================================================

% Vários exemplos de capitulos
% http://tug.ctan.org/info/MemoirChapStyles/
% https://texblog.org/2012/07/03/fancy-latex-chapter-styles/

% Para formatar todos os niveis de seção
\newcommand{\sectioningTopStyle}{\bfseries}
\newcommand{\sectioningSecStyle}{\bfseries}
\newcommand{\sectioningFinalDot}{} % Para poder adicionar um character no final da numeração

% Partes
% Declaração: Parte 1
\newcommand{\partAssertionFont}{\sffamily}
\newcommand{\partAssertionStyle}{\sectioningTopStyle}
\newcommand{\partAssertionSize}{\large}
% Título da parte
\newcommand{\partFont}{\sffamily}
\newcommand{\partStyle}{\sectioningTopStyle}
\newcommand{\partSize}{\Huge}
\definecolor{part_count}{RGB}{ 100, 100, 100 }
\definecolor{part_text}{RGB}{ 10, 70, 160 }

% Capítulos
% \newcommand{\chapterFont}{\sffamily}
\newcommand{\chapterFont}{\sffamily}
\newcommand{\chapterStyle}{\sectioningTopStyle}
\newcommand{\chapterSize}{\Huge}
\definecolor{chapter_label}{RGB}{ 100, 100, 100 }
\definecolor{chapter_count}{RGB}{ 100, 100, 100 }
\definecolor{chapter_text}{RGB}{ 10, 70, 160 }

% Seções
\newcommand{\sectionFont}{\sffamily}
\newcommand{\sectionStyle}{\sectioningSecStyle}
\newcommand{\sectionSize}{\Large}
\definecolor{section_count}{RGB}{ 100, 100, 100 }
\definecolor{section_text}{RGB}{ 10, 70, 160 }

% Sub Seções
\newcommand{\subsectionFont}{\sffamily}
\newcommand{\subsectionStyle}{\sectioningSecStyle}
\newcommand{\subsectionSize}{\large}
\definecolor{subsection_count}{RGB}{ 100, 100, 100 }
\definecolor{subsection_text}{RGB}{ 10, 70, 160 }

% Sub Sub Seções
\newcommand{\subsubsectionFont}{\sffamily}
\newcommand{\subsubsectionStyle}{\sectioningSecStyle}
\newcommand{\subsubsectionSize}{\normalsize}
\definecolor{subsubsection_count}{RGB}{ 100, 100, 100 }
\definecolor{subsubsection_text}{RGB}{ 10, 70, 160 }

% Sub Sub Sub Seções
\newcommand{\subsubsubsectionFont}{\sffamily}
\newcommand{\subsubsubsectionStyle}{\sectioningSecStyle}
\newcommand{\subsubsubsectionSize}{\normalsize}
\definecolor{subsubsubsection_count}{RGB}{ 100, 100, 100 }
\definecolor{subsubsubsection_text}{RGB}{ 10, 70, 160 }

\definecolor{paragraph_text}{RGB}{ 15, 15, 15 } % parágrafo de leis ou documentos
\definecolor{subparagraph_text}{RGB}{ 50, 50, 50 } % subparágrafo de leis ou documentos

% Corpo de texto
\definecolor{body_text}{RGB}{ 100, 100, 100 }
\definecolor{link_text}{RGB}{ 10, 70, 160 } % Links internos
\definecolor{cite_text}{RGB}{ 100, 100, 100 } % Links para citações
\definecolor{file_text}{RGB}{ 100, 100, 100 } % Links para arquivos
\definecolor{url_text}{RGB}{ 10, 70, 160 } % Links externos

% Sumário
\definecolor{summary_item}{RGB}{ 100, 100, 100 }
