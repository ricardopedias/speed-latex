% ---
% Capitulo que faz uso de elementos do glossario
% ---
\chapter{Orientações a respeito de glossários}

% ---
\section{Usar o glossário no texto}
% ---

Você pode definir as entradas do glossário no início do texto. Recomenda-se o
uso de um arquivo separado a ser inserido ainda no preâmbulo. Veja orientações
sobre inclusão de arquivos na \autoref{sec-include}.

No decorrer do texto, use os termos do glossário como na frase:

\begin{citacao}
Esta frase usa a palavra \gls{componente} e o plural de \glspl{filho}, ambas
definidas no glossário como filhas da entrada \gls{pai}. \Gls{equilibrio}
exemplifica o uso de um termo no início da frase. O software \gls{abntex2} é
escrito em \gls{latex}, que é definido no glossário como
\emph{`\glsdesc*{latex}'}.
\end{citacao}


A frase acima foi produzida com:

\begin{verbatim}
Esta frase usa a palavra \gls{componente} e o plural de \glspl{filho}, ambas
definidas no glossário como filhas da entrada \gls{pai}. \Gls{equilibrio}
exemplifica o uso de um termo no início da frase. O software \gls{abntex2} é
escrito em \gls{latex}, que é definido no glossário como
\emph{`\glsdesc*{latex}'}.
\end{verbatim}

Opcionalmente, incorpore todas as palavras do glossário de uma única vez ao
documento com o comando:

\begin{verbatim}
   \glsaddall
\end{verbatim}

A impressão efetiva do glossário é dada com:

\begin{verbatim}
   \printglossaries
\end{verbatim}

A impressão do glossário incorpora o número das páginas em que as entradas foram
citadas. Isso pode ser removido adicionando-se a opção \texttt{nonumberlist} em:

\begin{verbatim}
\usepackage[nonumberlist,style=index]{glossaries}%
\end{verbatim}

% ---
\section{Compilar um documento com glossário}
\label{sec-compilar-glossario}
% ---

Para compilar um documento \LaTeX\ com glossário use:

\begin{verbatim}
   pdflatex ARQUIVO_PRINCIPAL.tex
   bibtex ARQUIVO_PRINCIPAL.aux
   makeindex ARQUIVO_PRINCIPAL.idx
   makeindex ARQUIVO_PRINCIPAL.nlo -s nomencl.ist -o ARQUIVO_PRINCIPAL.nls
   makeglossaries ARQUIVO_PRINCIPAL.aux
   pdflatex ARQUIVO_PRINCIPAL.tex
   pdflatex ARQUIVO_PRINCIPAL.tex
\end{verbatim}

O comando \texttt{makeglossaries} é um aplicativo Perl instalado
automaticamente pelas distribuições MacTeX, TeX Live e MiKTeX. Geralmente
usuários de Linux e de Mac OS X já possuem o interpretador Perl\footnote{O Perl
é uma linguagem de programação de scripts muito utilizada pela comunidade de
software livre. Veja o site do projeto em \url{http://www.perl.org/}.} instalado
e configurado e nenhuma configuração adicional é necessária.

Usuários de Windows, por outro lado, precisam instalar a ferramenta Perl para
que seja possível usar \texttt{makeglossaries}. Por sorte isso é simples. Para
obter a instalação do Perl para seu sistema operacional visite \url{http://www.perl.org/get.html}.

Alternativamente ao aplicativo Perl \texttt{makeglossaries}, é possível usar o
aplicativo \texttt{makeglossariesgui}\footnote{O título do aplicativo no CTAN
é \textit{Java GUI alternative to makeglossarires script}.}, que possui uma
interface gráfica baseada em Java. Para isso, consulte
\url{http://www.ctan.org/pkg/makeglossariesgui}. Funciona em Windows,
Linux e Mac OS X.

% ---
\section{Configuração de glossários}
% ---

O pacote \textsf{glossaries}, usado na produção dos glossários deste exemplo,
possui diversas configurações. É possível alterar o estilo da impressão do
glossário, criar campos adicionais, usar diversos glossários em
arquivos separados. Para isso e outras informações, consulte a documentação do
pacote \textsf{glossaries}: \url{http://www.ctan.org/pkg/glossaries}.

Consulte também o livro da WikiBooks sobre a produção de glossários:
\url{http://en.wikibooks.org/wiki/LaTeX/Glossary}.


\subsection{Estilos do glossário}

O pacote \textsf{glossaries} traz dezenas de estilos pré-definidos de
glossários. Eles estão disponíveis no capítulo 15 do manual do pacote
\cite{talbot2012}. O capítulo 16 contém instruções sobre como criar um estilo
personalizado.

Os estilos podem ser alterados com:

\begin{verbatim}
   \setglossarystyle{altlisthypergroup}
\end{verbatim}

O estilo \texttt{index} é ideal para construção de glossários com diversos
níveis hierárquicos do tipo pai-filho. Já o modelo \texttt{altlisthypergroup} é
mais adequado para glossários sem hierarquias. Teste também o modelo
\texttt{tree}.

Se desejar um único estilo de glossário padrão no documento, alternativamente
inclua a opção \texttt{style} nas opções da classe, do
seguinte modo:

\begin{verbatim}
   \usepackage[style=index]{glossaries}
\end{verbatim}

% ---
\section{Problemas com a ordem das palavras?}
% ---

Este exemplo do \abnTeX\ utiliza a ferramenta \texttt{makeindex} -- padrão das
distribuições \LaTeX\ mais comuns -- para ordenar as entradas do glossário.
Porém, essa ferramenta não possui opções de \textit{collation} e não funciona
bem para palavras escritas em idiomas que não sejam inglês.
Por isso, pode acontecer que letras acentuadas e outros caracteres
internacionais sejam ordenados de forma incorreta, como no exemplo (palavras não
necessariamente presentes na língua portuguesa):

\begin{alineas}
 \item Amor: ...
 \item Aviar: ...
 \item Avião: ...
 \item Aço: ...
\end{alineas}

Por sorte, é possível substituir o uso do \texttt{makeindex}
pelo \texttt{xindy}\footnote{\url{http://www.xindy.org/}}. Para isso, faça o
seguinte:

\begin{alineas}
  \item Certifique-se de que o Xindy esteja instalado. Em um terminal, digite:
  \texttt{xindy --version}\footnote{Caso o Xindy não esteja presente no sistema, é necessário
    instalá-lo. Usuários Linux Debian/Ubuntu podem usar: \texttt{sudo
    apt-get install xindy}. Usuários Windows e Mac podem acessar a página do
    Xindy, baixá-lo e instalá-lo.};
  \item No código LaTeX, ainda no preâmbulo, inclua a seguinte opção ao pacote glossaries:
  \begin{verbatim}
  \usepackage[xindy={language=portuguese},nonumberlist=true]{glossaries}
  \end{verbatim}
  \item Compile o glossário normalmente, conforme a
  \autoref{sec-compilar-glossario}.
\end{alineas}
