
% ----------------------------------------------------------
% Introdução (exemplo de capítulo sem numeração, mas presente no Sumário)
% ----------------------------------------------------------
\chapter{Introdução}
% ----------------------------------------------------------

Este documento e seu código-fonte são exemplos de referência de uso do Speed Latex,
que faz uso da excelente classe \textsf{memoir} e da customização oferecida pelo projeto \textsf{\abnTeX}. Optou-se por utilizar a classe \textsf{\abnTeX} sem alterá-la, seguindo a ``recomendação dos autores do projeto''\footnote{\url{https://github.com/abntex/abntex2/wiki/ComoCustomizar}}.

\textbf{Atenção:} Embora faça-se uso do \textsf{\abnTeX}, o modelo "Livro Simples"\ não segue as regras da ABNT.

\section{Modelos disponíveis}

O Speed Latex disponibiliza vários modelos prontos para criação de documentos baseados em Latex.
Confira abaixo a lista de modelos disponíveis:

\begin{itemize}

    \item \textbf{abnt-academic-work} - Para Trabalhos Acadêmicos/TCC (ABNT);
    \item \textbf{abnt-article} - Para Artigos Científicos (ABNT);
    \item \textbf{abnt-research-project} - Para Projetos de Pesquisa (ABNT);
    \item \textbf{common-letter} - Para criação de Cartas;
    \item \textbf{common-document} - Para criação de Documentos;
    \item \textbf{common-book} - Para criação de Livros

\end{itemize}

\section{Organização}

Além de possibilitar a criação de projetos pré-organizados baseados em Latex, o Speed Latex
permite uma maior facilidade na personalização do documento. Isso é feito através de um arquivo de configuração existente para cada projeto, onde possível personalizar várias informações, como:

\begin{itemize}

    \item O conteúdo dos cabeçalhos e rodapés;
    \item As cores das seções e dos textos em geral;
    \item Definir informações padrões para uso no documento

\end{itemize}

\section{Atualizações}

O Speed Latex \footnote{\url{https://github.com/ricardopedias/speed-latex}} está em constante atualização. Sinta-se à vontade para conferir os releases \footnote{\url{https://github.com/ricardopedias/speed-latex/releases}} e acompanhar as atualizações.

Sinceramente espera-se que cada atualização do Speed Latex aprimore a qualidade do trabalho que você produzirá, de modo que o principal esforço seja concentrado no conteúdo produzido e não na formatação de documentos.
