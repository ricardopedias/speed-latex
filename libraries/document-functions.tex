%===================================================================================================
% Autor             : Ricardo Pereira Dias <rpdesignerfly@gmail.com>
% Data de criação   : 29 de Janeiro de 2019
% Referencias       : https://pt.overleaf.com/learn
%                     https://texblog.org/code-snippets/beamer-presentation/
%                     https://aprendolatex.wordpress.com/2007/06/09/seccoes-em-latex/
%                     https://pt.wikibooks.org/wiki/Latex/Se%C3%A7%C3%B5es
% --------------------------------------------------------------------------------------------------
% Este arquivo contém as funções personalizadas para facilitar a criação de documentos
%===================================================================================================

% Função para criar o capítulo
\newcommand{\docChapter}[1] {
    \chapter{
        \textcolor{chapter_text}{ #1 }
    }
}

% Função para criar seção de nivel 1
\newcommand{\docSection}[1] {
    \section{
        \textcolor{section_text}{ #1 }
    }
}

% Função para criar seção de nivel 2
\newcommand{\docSubSection}[1] {
    \subsection{
        \textcolor{subsection_text}{ #1 }
    }
}

% Função para criar seção de nivel 3
\newcommand{\docSubSubSection}[1] {
    \subsubsection{
        \textcolor{subsubsection_text}{ #1 }
    }
}

% Função para criar corpo de texto
\newcommand{\docBodyText}[1] {
    {\leavevmode\color{body_text}#1}
}

\newcommand{\link}[1] {\ignorespaces
    \textcolor{link_text}{#1}
    \ignorespaces
}

\newcommand{\textNormal}[1] {\ignorespaces
    \textcolor{body_text}{#1}
    \ignorespaces
}

\newcommand{\textSuccess}[1] {\ignorespaces
    \textcolor{success_text}{#1}
    \ignorespaces
}

\newcommand{\textDanger}[1] {\ignorespaces
    \textcolor{danger_text}{#1}
    \ignorespaces
}

\newcommand{\textInfo}[1] {\ignorespaces
    \textcolor{info_text}{#1}
    \ignorespaces
}

\newcommand{\textWarning}[1] {\ignorespaces
    \textcolor{warning_text}{#1}
    \ignorespaces
}

\newcommand{\textBold}[1] {\ignorespaces
    \textbf{#1}
    \ignorespaces
}

\newcommand{\textItalic}[1] {\ignorespaces
    \textit{#1}
    \ignorespaces
}

% Especiais

\newcommand{\textUnderline}[1] {\ignorespaces
    \uline{#1}
    \ignorespaces
}

\newcommand{\textUnderlineDanger}[1] {\ignorespaces
    {\color{danger_text}\uline{#1}}
    \ignorespaces
}

\newcommand{\textUnderlineSuccess}[1] {\ignorespaces
    {\color{success_text}\uline{#1}}
    \ignorespaces
}

\newcommand{\textDouble}[1] {\ignorespaces
    \uuline{#1}
    \ignorespaces
}

\newcommand{\textDoubleDanger}[1] {\ignorespaces
    {\color{danger_text}\uuline{#1}}
    \ignorespaces
}

\newcommand{\textDoubleSuccess}[1] {\ignorespaces
    {\color{success_text}\uuline{#1}}
    \ignorespaces
}

\newcommand{\textWave}[1] {\ignorespaces
    \uwave{#1}
    \ignorespaces
}

\newcommand{\textWaveDanger}[1] {\ignorespaces
    {\color{danger_text}\uwave{#1}}
    \ignorespaces
}

\newcommand{\textWaveSuccess}[1] {\ignorespaces
    {\color{success_text}\uwave{#1}}
    \ignorespaces
}

\newcommand{\textDelete}[1] {\ignorespaces
    \sout{#1}
    \ignorespaces
}

\newcommand{\textDeleteDanger}[1] {\ignorespaces
    {\color{danger_text}\sout{#1}}
    \ignorespaces
}

\newcommand{\textDeleteSuccess}[1] {\ignorespaces
    {\color{success_text}\sout{#1}}
    \ignorespaces
}

\newcommand{\textStriped}[1] {\ignorespaces
    \xout{#1}
    \ignorespaces
}

\newcommand{\textStripedDanger}[1] {\ignorespaces
    {\color{danger_text}\xout{#1}}
    \ignorespaces
}

\newcommand{\textStripedSuccess}[1] {\ignorespaces
    {\color{success_text}\xout{#1}}
    \ignorespaces
}

% \newcommand{\reduline}[1]{{\color{red}\uline{{\color{black}#1}}}}







\newcommand{\titulo}[1]
{
    \begin{center}
        \begin{minipage}[t]{10cm}
            \begin{center}
                {{\Large #1}}
            \end{center}
        \end{minipage}
    \end{center}
}

\newcommand{\autor}[1]
{
    \relax\vskip 0.2cm\noindent
    \begin{center}
        \begin{minipage}[t]{10cm}
            \begin{center}
                {{\large #1}}
            \end{center}
        \end{minipage}
    \end{center}
}

\newcommand{\doctipo}[1]
{
    \relax\vskip 0.2cm\noindent
    \begin{center}
        \begin{minipage}[t]{10cm}
            \begin{center}
                {{#1}} %\quad {(#2)}
            \end{center}
        \end{minipage}
    \end{center}
}

\newcommand{\orientacao}[1]
{
    \vskip 2cm\noindent
    \begin{center}
        {Orienta\c{c}\~ao: #1}
    \end{center}
}

\newcommand{\docarea}[1]
{
    \vskip 0.2cm\noindent
    \begin{center}
        {\'{A}rea de Concentra\c{c}\~ao: #1}
    \end{center}
}

\newcommand{\textofree}[1]
{
    %\vskip 1cm
    \noindent
    \begin{center}
        {#1}
    \end{center}
}
